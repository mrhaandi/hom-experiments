\documentclass[10pt,a4paper]{article}
\usepackage[utf8]{inputenc}
\usepackage[T1]{fontenc}
\usepackage[english]{babel}
\usepackage{amsmath}
\usepackage{amsfonts}
\usepackage{amssymb}
\usepackage{amsthm}
\usepackage{bussproofs}
\usepackage{graphicx}
\usepackage{lmodern}
\usepackage{microtype}
\usepackage{csquotes}
\usepackage{xcolor}
\usepackage{hyperref}

\usepackage[style=alphabetic,backend=bibtex]{biblatex}
\addbibresource{bibliography.bib}

\theoremstyle{plain}% default
\newtheorem{theorem}{Theorem}
\newtheorem{definition}[theorem]{Definition}
\newtheorem{lemma}[theorem]{Lemma}
\newtheorem{corollary}[theorem]{Corollary}
\newtheorem{problem}[theorem]{Problem}
\newtheorem{example}[theorem]{Example}
\newtheorem{proposition}[theorem]{Proposition}
\newtheorem{remark}[theorem]{Remark}

\title{Notes on Higher-Order Matching}
\author{Andrej Dudenhefner \and Aleksy Schubert}
\date{\today}

\begin{document}

\maketitle

\section{Bounded Parallel Intersection Type System}

We denote \emph{intersection types} by $A, B$, finite sets of intersection types by $\sigma, \tau$, and \emph{type atoms} by $a, b$.

\begin{definition}[Intersection Types]
\label{def:itypes}
$\begin{array}{rcl}
A, B &::=& a \mid \sigma \to A\\
\sigma, \tau &::=& \{A_1, \ldots, A_n\} \text{ where } n \geq 0
\end{array}$
\end{definition}

An \emph{environment}, denoted by $\Gamma$, is a finite set of \emph{type assumptions} having the shape $x : \sigma$ for distinct term variables.
\begin{definition}[Environment]
$\Gamma ::=\{x_1 : \sigma_1, \ldots, x_n : \sigma_n\}$ where $x_i \neq x_j$ for $i \neq j$.
\end{definition}
We may extend an environment $\Gamma$ by an additional assumption $x : \sigma$, written $\Gamma, x : \sigma$, where $x$ does not appear in any assumption in $\Gamma$.

We denote vectors of types or environments by $\bar{A}$, $\bar{\sigma}$, or $\bar{\Gamma}$ respectively.
If $\bar{\sigma} = (\sigma_1, \ldots, \sigma_n)$ and $\bar{A} = (A_1, \ldots, A_n)$, then $\sigma \Rightarrow A = (\sigma_1 \to A_1, \ldots, \sigma_n \to A_n)$.

Borrowing notation~\cite{DudenhefnerU21} we define type vector transformations as follows.

\begin{definition}[Type Vector Transformation]
Let $f : \{1, \ldots, n\} \to \{1, \ldots, m\}$ be a surjective function.
\begin{itemize}
\item $f(A_1, \ldots, A_n) = (C_1, \ldots, C_m)$ such that $C_i = \{ A_j \mid f(j) = i \}$
\item $f^{-1}(B_1, \ldots, B_m) = (D_1, \ldots, D_n)$ such that $D_i = B_{f(i)}$
\end{itemize}
\end{definition}

We tacitly extend the definition to environments where set formation is taken pointwise.

\begin{definition}[Parallel Intersection Type System]
\label{def:type-system}
~\\

\begin{minipage}{\textwidth}
\centering
\begin{tabular}{c}
{\RightLabel{\textnormal{(Ax)}}
\AxiomC{}
\UnaryInfC{$\{x : \bar{A}\} \vdash x : \bar{A}$}
\DisplayProof} \quad
{\RightLabel{\textnormal{($\Rightarrow$I)}}
\AxiomC{$\bar{\Gamma}, x: \bar{\sigma} \vdash t : \vec{A}$}
\UnaryInfC{$\bar{\Gamma} \vdash \lambda x.t : \bar{\sigma} \Rightarrow \bar{A}$}
\DisplayProof}\\\\
{\RightLabel{\textnormal{($\Rightarrow$E)}}
\AxiomC{$\bar{\Gamma} \vdash t : f(\bar{A}) \Rightarrow \bar{B}$}
\AxiomC{$\bar{\Delta} \vdash u : \bar{A}$}
\BinaryInfC{$\bar{\Gamma} \cup f(\bar{\Delta}) \vdash t \; u : \bar{B}$}
\DisplayProof}
\end{tabular}
\end{minipage}
\end{definition}

\begin{definition}[Dimensional Restriction]
$\bar{\Gamma} \vdash_k t : \bar{A}$ means that the length of vectors in some derivation is at most $k$.
\end{definition}

\begin{theorem}
$\Gamma \vdash_{\textnormal{BCD}} t : A$ iff for some $k$ we have $(\Gamma) \vdash_k t : (A)$.
\end{theorem}

\begin{theorem}
Given $\Gamma$, $A$, and $k$ it is decidable whether $(\Gamma) \vdash_k t : (A)$ holds.
\end{theorem}

\newpage

\printbibliography
\end{document}