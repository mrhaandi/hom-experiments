\documentclass[10pt,a4paper]{article}
\usepackage[utf8]{inputenc}
\usepackage[T1]{fontenc}
\usepackage[english]{babel}
\usepackage{amsmath}
\usepackage{amsfonts}
\usepackage{amssymb}
\usepackage{amsthm}
\usepackage{bussproofs}
\usepackage{graphicx}
\usepackage{lmodern}
\usepackage{microtype}
\usepackage{csquotes}
\usepackage{xcolor}
\usepackage{hyperref}

\usepackage[style=alphabetic,backend=bibtex]{biblatex}
\addbibresource{bibliography.bib}

\theoremstyle{plain}% default
\newtheorem{theorem}{Theorem}
\newtheorem{definition}[theorem]{Definition}
\newtheorem{lemma}[theorem]{Lemma}
\newtheorem{corollary}[theorem]{Corollary}
\newtheorem{problem}[theorem]{Problem}
\newtheorem{example}[theorem]{Example}
\newtheorem{proposition}[theorem]{Proposition}
\newtheorem{remark}[theorem]{Remark}

\title{Notes on Higher-Order Matching}
\author{Andrej Dudenhefner \and Aleksy Schubert}
\date{\today}

\begin{document}

\maketitle

\section{Bounded Parallel Intersection Type System}

We denote \emph{intersection types} by $A, B$, finite sets of intersection types by $\sigma, \tau$, and \emph{type atoms} by $a, b$.

\begin{definition}[Intersection Types]
\label{def:itypes}
$\begin{array}{rcl}
A, B &::=& a \mid \sigma \to A\\
\sigma, \tau &::=& \{A_1, \ldots, A_n\} \text{ where } n \geq 0
\end{array}$
\end{definition}

An \emph{environment}, denoted by $\Gamma$, is a finite set of \emph{type assumptions} having the shape $x : \sigma$ for distinct term variables.
\begin{definition}[Environment]
$\Gamma ::=\{x_1 : \sigma_1, \ldots, x_n : \sigma_n\}$ where $x_i \neq x_j$ for $i \neq j$.
\end{definition}
We may extend an environment $\Gamma$ by an additional assumption $x : \sigma$, written $\Gamma, x : \sigma$, where $x$ does not appear in any assumption in $\Gamma$.

We denote vectors of types and sets of types by $\bar{A}$ and $\bar{\sigma}$ respectively.
If $\bar{\sigma} = (\sigma_1, \ldots, \sigma_n)$ and $\bar{A} = (A_1, \ldots, A_n)$, then $\sigma \Rightarrow A = (\sigma_1 \to A_1, \ldots, \sigma_n \to A_n)$.

Extending notation~\cite{DudenhefnerU21} from surjective functions to binary relations we define type vector transformations as follows.

\begin{definition}[Type Vector Transformation]
Let $R \subseteq \{1, \ldots, n\} \times \{1, \ldots, m\}$ be a left-total and right-total binary relation.
\begin{itemize}
\item $R(A_1, \ldots, A_n) := (\tau_1, \ldots, \tau_m)$ where $\tau_i = \{A_j \mid j \, R\, i\}$
\item $R(\sigma_1, \ldots, \sigma_n) := (\tau_1, \ldots, \tau_m)$ where $\tau_i = \bigcup \{\sigma_j \mid j \, R\, i\}$
\item $R(\{x_1 : \sigma_1, \ldots, x_l : \sigma_l\}) := \{x_1 : R(\sigma_1), \ldots, x_l : R(\sigma_l)\}$
\end{itemize}
\end{definition}

\begin{example}
For $R = \{(1, 1), (1, 3), (2, 1), (3, 2), (3, 3)\}$ we have \[R(a, b, c) = (\{a, b\}, \{c\}, \{a, c\})\]
\end{example}

\begin{definition}[Parallel Intersection Type System]
\label{def:type-system}
~\\

\begin{minipage}{\textwidth}
\centering
\begin{tabular}{l}
{\RightLabel{\textnormal{(Ax)}}
\AxiomC{}
\UnaryInfC{$\{x : \bar{A}\} \vdash x : \bar{A}$}
\DisplayProof} \quad
{\RightLabel{\textnormal{($\omega$)}}
\AxiomC{}
\UnaryInfC{$\emptyset \vdash t : ()$}
\DisplayProof}\\\\
{\RightLabel{\textnormal{($\Rightarrow$I)}}
\AxiomC{$\Gamma, x: \bar{\sigma} \vdash t : \vec{A}$}
\UnaryInfC{$\Gamma \vdash \lambda x.t : \bar{\sigma} \Rightarrow \bar{A}$}
\DisplayProof}\quad
{\RightLabel{\textnormal{($\Rightarrow$E)}}
\AxiomC{$\Gamma \vdash t : R(\bar{A}) \Rightarrow \bar{B}$}
\AxiomC{$\Delta \vdash u : \bar{A}$}
\BinaryInfC{$\Gamma \cup R(\Delta) \vdash t \; u : \bar{B}$}
\DisplayProof}
\end{tabular}
\end{minipage}
\end{definition}

\begin{example}~\\
Let 
\begin{itemize}
\item $\Gamma_1 = \{x : (\{b, c\} \to a)\}$
\item $\Gamma_2 = \{y : (a \to b, a \to c), z : (a, a)\}$
\item $R_1 = \{(1,1), (2,1)\}$
\item $R_2 = \{(1,1), (1,2)\}$
\end{itemize}
We have the following derivation


\AxiomC{$\Gamma_1 \vdash x : R_1(b, c) \Rightarrow (a)$}
\AxiomC{$\{y : (a \to b, a \to c)\} \vdash y : R_2(a) \Rightarrow (b, c)$}
\AxiomC{$\{z : (a)\} \vdash z : (a)$}
\BinaryInfC{$\Gamma_2 \vdash y\,z : (b, c)$}
\BinaryInfC{$\Gamma_1 \cup R_1(\Gamma_2) \vdash x\,(y\,z) : (a)$}
\DisplayProof
\end{example}


\begin{definition}[Dimensional Restriction]
$\Gamma \vdash_k t : \bar{A}$ means that the length of vectors in some derivation of $\Gamma \vdash t : \bar{A}$ is at most $k$.
\end{definition}

\begin{theorem}
$\Gamma \vdash_{\textnormal{CDV}} t : A$ iff for some $k$ we have $\Gamma \vdash_k t : (A)$.
\end{theorem}

\begin{proposition}
Given $\Gamma$, $\bar{A}$, and $k$ it is decidable whether $\Gamma \vdash_k t : \bar{A}$ holds for some $t$.
\end{proposition}

\begin{proof}[(Proof Idea)]
In order to decide whether $\Gamma \vdash_k t : \bar{A}$ holds for some $t$, it suffices to consider $t$ in $\beta$-normal form.
For an \textsf{ASPACE} decision algorithm we need to limit the effective number of variables in type environments.
Assume $x : \bar{\sigma}$ occurs in some environment for an inhabitant in $\beta$-normal form.
Then each type $B$ which occurs in each set in each component of $\bar{\sigma}$ is a subformula of some type occurring in $\Gamma$ or $A$.
Therefore, there is finitely many distinct $\bar{\sigma}$.
By the pigeonhole principle, if there are more variables in a type environment, some must have identical types and thus a redundant copy can be removed.
Therefore, there are finitely many environements and types to consider throughout the cesition procedure.
\end{proof}

\begin{remark}
For fourth order matching $k$ is the product for the number of fragments of the right-hand sides.
However, intersection types in $\Gamma$ come from subject expansion of an intersection typing for an arbitrary candidate solution.
Therefore, not a sufficient restriction (at first glance).
However, one can try to relate types in $\Gamma$ to given simple types in the given fourth-order matching problem, possibly obtaining a finite bound of occurring types.
\end{remark}

\begin{definition}[Refinement]
An intersection type $A$ refines a simple type $\varphi$, written $A \prec \varphi$ if
\begin{itemize}
\item $A$ is a constant and $\varphi = \bullet$
\item $A = \sigma \to B$ and $\varphi = \varphi_1 \to \varphi_2$ and $B \prec \varphi_2$ and for all $A_1 \in \sigma$ we have $A_1 \prec \varphi_1$ 
\end{itemize}
\end{definition}

\begin{definition}[Shift]
We consider type atoms as words of natural numbers.
\begin{itemize}
\item ${\downarrow_i}w = wi$
\item ${\downarrow_i}(\{A_1, \ldots, A_n\} \to A) = (\{{\downarrow_i}A_1, \ldots, {\downarrow_i}A_n\} \to {\downarrow_i}A)$
\end{itemize}
\end{definition}

For example ${\downarrow_2}(\{12, 3\} \to \varepsilon) = \{122, 32\} \to 2$.

\begin{lemma}
If $r$ is in $\beta$-normal form and $\Phi \vdash_{\textnormal{STLC}} r : \varphi$ is a $\eta$-long derivation, then there exists $\Gamma$, $A$ such that
\begin{enumerate}
\item $\Gamma \prec \Phi$ and $A \prec \varphi$
\item $\Gamma \vdash r : A$
\item for all $t$ such that $\Phi \vdash_{\textnormal{STLC}} t : \varphi$ is an $\eta$-long derivation (not necessarily $\beta$-normal) and $\Gamma \vdash t : A$, then $t \twoheadrightarrow_\beta r$
\end{enumerate}
\end{lemma}

\begin{proof}[(Proof Sketch)]
First we focus on $\beta$-normal forms.
The crucial step application.
Consider the example term $x\,r_1\,r_2$. By inversion we have $(x : \varphi_1 \to \varphi_2 \to \bullet) \in \Phi$ and $\Phi \vdash_{\textnormal{STLC}} r_i : \varphi_i$.
By induction hypothesis there are $\Gamma_1, A_1, \Gamma_2, A_2$ satisfying the above conditions.
We set $\Gamma = (x : {\downarrow_1} A_1 \to {\downarrow_2} A_2 \to \varepsilon) \cup {\downarrow_1}\Gamma_1 \cup {\downarrow_2}\Gamma_2$
and $A = \varepsilon$.
\end{proof}

\newpage

\printbibliography
\end{document}