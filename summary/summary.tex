\documentclass[10pt,a4paper]{article}
\usepackage[utf8]{inputenc}
\usepackage[T1]{fontenc}
\usepackage[english]{babel}
\usepackage{amsmath}
\usepackage{amsfonts}
\usepackage{amssymb}
\usepackage{amsthm}
\usepackage{bussproofs}
\usepackage{graphicx}
\usepackage{lmodern}
\usepackage{microtype}
\usepackage{csquotes}
\usepackage{xcolor}
\usepackage{hyperref}

\usepackage[style=alphabetic,backend=bibtex]{biblatex}
\addbibresource{bibliography.bib}

\theoremstyle{plain}% default
\newtheorem{theorem}{Theorem}
\newtheorem{definition}[theorem]{Definition}
\newtheorem{lemma}[theorem]{Lemma}
\newtheorem{corollary}[theorem]{Corollary}
\newtheorem{problem}[theorem]{Problem}
\newtheorem{example}[theorem]{Example}
\newtheorem{proposition}[theorem]{Proposition}
\newtheorem{remark}[theorem]{Remark}

\title{Notes on Higher-Order Matching}
\author{Andrej Dudenhefner \and Aleksy Schubert}
\date{\today}

\begin{document}

\maketitle

\section{Bounded Parallel Intersection Type System}

We denote \emph{intersection types} by $A, B$, finite sets of intersection types by $\sigma, \tau$, and \emph{type atoms} by $a, b$.

\begin{definition}[Intersection Types]
\label{def:itypes}
$\begin{array}{rcl}
A, B &::=& a \mid \sigma \to A\\
\sigma, \tau &::=& \{A_1, \ldots, A_n\} \text{ where } n \geq 0
\end{array}$
\end{definition}

An \emph{environment}, denoted by $\Gamma$, is a finite set of \emph{type assumptions} having the shape $x : \sigma$ for distinct term variables.
\begin{definition}[Environment]
$\Gamma ::=\{x_1 : \sigma_1, \ldots, x_n : \sigma_n\}$ where $x_i \neq x_j$ for $i \neq j$.
\end{definition}
We may extend an environment $\Gamma$ by an additional assumption $x : \sigma$, written $\Gamma, x : \sigma$, where $x$ does not appear in any assumption in $\Gamma$.

We denote vectors of types and sets of types by $\bar{A}$ and $\bar{\sigma}$ respectively.
If $\bar{\sigma} = (\sigma_1, \ldots, \sigma_n)$ and $\bar{A} = (A_1, \ldots, A_n)$, then $\sigma \Rightarrow A = (\sigma_1 \to A_1, \ldots, \sigma_n \to A_n)$.

Extending notation~\cite{DudenhefnerU21} from surjective functions to binary relations we define type vector transformations as follows.

\begin{definition}[Type Vector Transformation]
Let $R \subseteq \{1, \ldots, n\} \times \{1, \ldots, m\}$ be a left-total and right-total binary relation.
\begin{itemize}
\item $R(A_1, \ldots, A_n) := (\tau_1, \ldots, \tau_m)$ where $\tau_i = \{A_j \mid j \, R\, i\}$
\item $R(\sigma_1, \ldots, \sigma_n) := (\tau_1, \ldots, \tau_m)$ where $\tau_i = \bigcup \{\sigma_j \mid j \, R\, i\}$
\item $R(\{x_1 : \sigma_1, \ldots, x_l : \sigma_l\}) := \{x_1 : R(\sigma_1), \ldots, x_l : R(\sigma_l)\}$
\end{itemize}
\end{definition}

\begin{example}
For $R = \{(1, 1), (1, 3), (2, 1), (3, 2), (3, 3)\}$ we have \[R(a, b, c) = (\{a, b\}, \{c\}, \{a, c\})\]
\end{example}

\begin{definition}[Parallel Intersection Type System]
\label{def:type-system}
~\\

\begin{minipage}{\textwidth}
\centering
\begin{tabular}{l}
{\RightLabel{\textnormal{(Ax)}}
\AxiomC{}
\UnaryInfC{$\{x : \bar{A}\} \vdash x : \bar{A}$}
\DisplayProof} \quad
{\RightLabel{\textnormal{($\omega$)}}
\AxiomC{}
\UnaryInfC{$\emptyset \vdash t : ()$}
\DisplayProof}\\\\
{\RightLabel{\textnormal{($\Rightarrow$I)}}
\AxiomC{$\Gamma, x: \bar{\sigma} \vdash t : \vec{A}$}
\UnaryInfC{$\Gamma \vdash \lambda x.t : \bar{\sigma} \Rightarrow \bar{A}$}
\DisplayProof}\quad
{\RightLabel{\textnormal{($\Rightarrow$E)}}
\AxiomC{$\Gamma \vdash t : R(\bar{A}) \Rightarrow \bar{B}$}
\AxiomC{$\Delta \vdash u : \bar{A}$}
\BinaryInfC{$\Gamma \cup R(\Delta) \vdash t \; u : \bar{B}$}
\DisplayProof}
\end{tabular}
\end{minipage}
\end{definition}

Similar systems:
\begin{itemize}
\item The system $\lambda^{\mbox{CIL}}$ by Wells et al
  \cite{WellsDMT02} (see \cite{WellsH02} for a simplified version of
  the system) is a system in Church style with explicit type
  annotations in $\lambda$-biders where different derivations for the
  same subterm are managed by means of records. A record of types with
  labels from $I$ is derived here by a record of terms of
  corresponding labels from $I$.
\item The proof system ISL presented by Pimentel, Ronchi Della Rocca
  and Roversi \cite{PimentelRR12} features independent derivations in
  tuples led according to the same proof rules. Effectively this works
  like an intersection type system with vectors. However, the system
  does not have proof terms. A version of the proof system formulated
  in terms of a sequent calculus is presented in \cite{RoccaSSV10}.
\end{itemize}

Dissimilar systems:
\begin{itemize}
\item The system by Luigi Liquori and Simona Ronchi della Rocca
  \cite{LiquoriR07} is a system with explicit annotations at
  $\lambda$-binders. It introduces  terms in which variables in
  binders are annotadet with indices. Then these indices serve as
  pointers to binders in derivations, and the binders in derivations
  contain types.
\end{itemize}


\begin{example}~\\
Let 
\begin{itemize}
\item $\Gamma_1 = \{x : (\{b, c\} \to a)\}$
\item $\Gamma_2 = \{y : (a \to b, a \to c), z : (a, a)\}$
\item $R_1 = \{(1,1), (2,1)\}$
\item $R_2 = \{(1,1), (1,2)\}$
\end{itemize}
We have the following derivation


\AxiomC{$\Gamma_1 \vdash x : R_1(b, c) \Rightarrow (a)$}
\AxiomC{$\{y : (a \to b, a \to c)\} \vdash y : R_2(a) \Rightarrow (b, c)$}
\AxiomC{$\{z : (a)\} \vdash z : (a)$}
\BinaryInfC{$\Gamma_2 \vdash y\,z : (b, c)$}
\BinaryInfC{$\Gamma_1 \cup R_1(\Gamma_2) \vdash x\,(y\,z) : (a)$}
\DisplayProof
\end{example}


\begin{definition}[Dimensional Restriction]
$\Gamma \vdash_k t : \bar{A}$ means that the length of vectors in some derivation of $\Gamma \vdash t : \bar{A}$ is at most $k$.
\end{definition}

\begin{theorem}
$\Gamma \vdash_{\textnormal{CDV}} t : A$ iff for some $k$ we have $\Gamma \vdash_k t : (A)$.
\end{theorem}

\begin{proposition}
Given $\Gamma$, $\bar{A}$, and $k$ it is decidable whether $\Gamma \vdash_k t : \bar{A}$ holds for some $t$.
\end{proposition}

\begin{proof}[(Proof Idea)]
In order to decide whether $\Gamma \vdash_k t : \bar{A}$ holds for some $t$, it suffices to consider $t$ in $\beta$-normal form.
For an \textsf{ASPACE} decision algorithm we need to limit the effective number of variables in type environments.
Assume $x : \bar{\sigma}$ occurs in some environment for an inhabitant in $\beta$-normal form.
Then each type $B$ which occurs in each set in each component of $\bar{\sigma}$ is a subformula of some type occurring in $\Gamma$ or $A$.
Therefore, there is finitely many distinct $\bar{\sigma}$.
By the pigeonhole principle, if there are more variables in a type environment, some must have identical types and thus a redundant copy can be removed.
Therefore, there are finitely many environements and types to consider throughout the cesition procedure.
\end{proof}

\begin{remark}
For fourth order matching $k$ is the product for the number of fragments of the right-hand sides.
However, intersection types in $\Gamma$ come from subject expansion of an intersection typing for an arbitrary candidate solution.
Therefore, not a sufficient restriction (at first glance).
However, one can try to relate types in $\Gamma$ to given simple types in the given fourth-order matching problem, possibly obtaining a finite bound of occurring types.
\end{remark}

\begin{definition}[Refinement]
An intersection type $A$ refines a simple type $\varphi$, written $A \prec \varphi$ if
\begin{itemize}
\item $A$ is a constant and $\varphi = \bullet$
\item $A = \sigma \to B$ and $\varphi = \varphi_1 \to \varphi_2$ and $B \prec \varphi_2$ and for all $A_1 \in \sigma$ we have $A_1 \prec \varphi_1$ 
\end{itemize}
\end{definition}

\begin{definition}[Shift]
We consider type atoms as words of natural numbers.
\begin{itemize}
\item ${\downarrow_i}w = wi$
\item ${\downarrow_i}(\{A_1, \ldots, A_n\} \to A) = (\{{\downarrow_i}A_1, \ldots, {\downarrow_i}A_n\} \to {\downarrow_i}A)$
\end{itemize}
\end{definition}

For example ${\downarrow_2}(\{12, 3\} \to \varepsilon) = \{122, 32\} \to 2$.

\begin{lemma}[Equality Characterization]
If $r$ is in $\beta$-normal form and $\Phi \vdash_{\textnormal{STLC}} r : \varphi$ is a $\eta$-long derivation, then there exists $\Gamma$, $A$ such that
\begin{enumerate}
\item $\Gamma \prec \Phi$ and $A \prec \varphi$
\item $\Gamma \vdash r : A$
\item for all $t$ such that $\Phi \vdash_{\textnormal{STLC}} t : \varphi$ is an $\eta$-long derivation (not necessarily $\beta$-normal) and $\Gamma \vdash t : A$, then $t \twoheadrightarrow_\beta r$
\end{enumerate}
\end{lemma}

\begin{proof}[(Proof Idea)]
First we focus on $\beta$-normal forms.
The crucial step application.
Consider the example term $r = x\,r_1\,r_2$. By inversion we have $(x : \varphi_1 \to \varphi_2 \to \bullet) \in \Phi$ and $\Phi \vdash_{\textnormal{STLC}} r_i : \varphi_i$.
By induction hypothesis there are $\Gamma_1, A_1, \Gamma_2, A_2$ satisfying the above conditions.
We set $\Gamma = (x : {\downarrow_1} A_1 \to {\downarrow_2} A_2 \to \varepsilon) \cup {\downarrow_1}\Gamma_1 \cup {\downarrow_2}\Gamma_2$
and $A = \varepsilon$.
\end{proof}

\begin{lemma}[Disequality Characterization]
If $r$ is in $\beta$-normal form and $\Phi \vdash_{\textnormal{STLC}} r : \varphi$ is a $\eta$-long derivation, then there exists $\Gamma$, $A$ such that
\begin{enumerate}
\item $\Gamma \prec \Phi$ and $A \prec \varphi$
\item $\Gamma \vdash r : A$
\item for all $t$ such that $\Phi \vdash_{\textnormal{STLC}} t : \varphi$ is an $\eta$-long derivation (not necessarily $\beta$-normal) and $\Gamma \vdash t : A$, then $t \not\twoheadrightarrow_\beta r$
\end{enumerate}
\end{lemma}

\begin{example}
Consider $\Phi = \{g : \bullet \to \bullet \to \bullet, y : \bullet\}$ and $r = g\,y\,y$.
Terms $t$ in $\beta$-normal form with an $\eta$-long derivation of $\Phi \vdash t : \bullet$ such that $t \neq r$ are
\begin{enumerate}
\item $y$\\
This can be characterized by $\{y : \{\varepsilon\}\} \vdash y : \varepsilon$
\item $g\,t'\,t''$ such that $t' \neq y$, which means $t = g\,t_1\,t_2$.\\
This can be characterized by $\{g : \{{\downarrow_1}\{\varepsilon\} \to \{\} \to \varepsilon\}, \{\} \to \{\} \to {\downarrow_1}\{\varepsilon\} \} \vdash g\,t'\,t'' : \varepsilon$
\item $g\,t''\,t'$ such that $t' \neq y$, which means $t = g\,t_1\,t_2$.\\
This can be characterized by $\{g : \{\{\} \to {\downarrow_2}\{\varepsilon\} \to \varepsilon\}, \{\} \to \{\} \to {\downarrow_2}\{\varepsilon\} \} \vdash g\,t''\,t' : \varepsilon$
\end{enumerate}
Putting it all together we can take $\Gamma = \{y : \{\varepsilon\}, g : \{{\downarrow_1}\{\varepsilon\} \to \{\} \to \varepsilon, \{\} \to \{\} \to {\downarrow_1}\{\varepsilon\}, \{\} \to {\downarrow_2}\{\varepsilon\} \to \varepsilon\}, \{\} \to \{\} \to {\downarrow_2}\{\varepsilon\}\}$ and $A = \varepsilon$ such that $\Gamma \vdash t : A$.
\end{example}

\begin{example}
Consider $\Phi = \{f : (\bullet \to \bullet) \to \bullet, g : \bullet \to \bullet \to \bullet, y : \bullet\}$ and $r = f\,(\lambda x.g\,x\,y)$.
This extends the previos example.
Terms $t$ in $\beta$-normal form with an $\eta$-long derivation of $\Phi \vdash t : \bullet$ are
\begin{enumerate}
\item $y$\\
This can be characterized by $\{y : \{\varepsilon\}\} \vdash y : \varepsilon$
\item $g\,t_1\,t_2$\\
This can be characterized by $\{g : \{\} \to \{\} \to \varepsilon \} \vdash g\,t_1\,t_2 : \varepsilon$
\item $f\,(\lambda x.t')$ such that $t' \neq g\,x\,y$\\
Here we have $\Phi, x : \bullet : g\,x\,y : \bullet$ and can derive $\Gamma'$, $A' = \varepsilon$ as characterization for any $\beta$-normal $\eta$-long typed term different from $g\,x\,y$.
Let $\Gamma'' = \Gamma' \setminus \{x\}$
Overall we can take $\Gamma = \Gamma'' \cup \{y : \{\varepsilon\}, g : \{\} \to \{\} \to \varepsilon, f : \Gamma'(x) \to \varepsilon\}$ and $A = \varepsilon$.
\end{enumerate}
\end{example}

\begin{lemma}
\label{lem:transform}
If $\Gamma \vdash t : \bar{A}$ and $R$ is right-unique, then $R(\Gamma) \vdash t : R(\bar{A})$.
\end{lemma}

\begin{lemma}
If in a derivation of the judgment $\Gamma \vdash_k t : \bar{A}$ there occurs $\Delta \vdash_k u : \bar{B}$ such that for some distinct $i, j$
$B_i = B_j$ and $\forall x.\Delta(x)_i = \Delta(x)$, then there is a derivation of $\Gamma \vdash_k t : \bar{A}$ that has no such property (TODO rephrase).
\end{lemma}

\begin{proof}
Use Lemma~\ref{lem:transform} to collapse redundant entries and idempotence of intersection.
\end{proof}

\newpage

\section{Application to Fourth-order Matching}


\newpage

\section*{Open Questions}
\begin{remark}[Open Questions]
\begin{itemize}
\item Relationship to decidability in each finite rank\\
\url{https://dl.acm.org/doi/10.1145/317636.317788}\\
\url{https://www.sciencedirect.com/science/article/pii/S1571066105050656}\\
\url{https://www.sciencedirect.com/science/article/pii/S0304397503005772}
%Sébastien Carlier and J. B. Wells. Expansion: the crucial mechanism for type inference with intersection types: A survey and explanation. In Proc. 3rd Int'l Workshop Intersection Types & Related Systems (ITRS 2004), pages 173-202, 2005. The ITRS '04 proceedings appears as vol. 136 (2005-07-19) of Elec. Notes in Theoret. Comp. Sci. (PDF)
%Assaf J. Kfoury and J. B. Wells. Principality and type inference for intersection types using expansion variables. Theoret. Comput. Sci., 311(1-3):1-70, 2004. Supersedes [17]. For omitted proofs, see the longer report [39]. (PDF)
\item Church-style presentation of the system
\item Exact compexity of inhabitation wrt. dimension
\item Relationship to the multiset bounded intersection system (Jakob and Andrej)
\item Relationship of relation R to expansion variables
\end{itemize}
\end{remark}


\newpage

\printbibliography
\end{document}
%%% Local Variables:
%%% mode: latex
%%% TeX-master: t
%%% End:
